\section{Training an \ac{RNN} for \ac{ASR}}\label{ds}
As stated above, this project does not aim at training a state of the art \ac{STT} engine. Because the \ac{SW} algorithm used for local alignment is tolerant to a certain amount of errors in the transcripts, the \ac{RNN} need only be \textit{good enough} for the task at hand (\ac{FA}). If such a network can be trained under the given circumstances it could be used in the \ac{ASR} stage of the pipeline. The pipeline would then become become self-contained and would not ne dependent on a commercial solution that cannot eb tuned and whose inner workings are unknown. Furthermore such a \ac{RNN} would open up to recognizing languages for which there is not a third-party solution yet, such as Swiss German.

\subsection{Exploiting the \textit{DeepSpeech} model}

A \ac{NN} that had quite an impact on \ac{ASR} was \textit{DeepSpeech} \parencite{deepspeech} which reached recognition rates near-par to human performance, despite using a comparably simpler than traditional speech systems. Because the relation between audio signal and text was learned \ac{E2E} the network was also pretty robust to distortions like background noise or speaker variation. An open source implementation of a DeepSpeech model is available from Mozilla \footnote{\url{https://github.com/mozilla/DeepSpeech}}. Since this implementation uses a \ac{LM}, the quality of the model is measured as the percentage of misspelled or wrong words (called \ac{WER}) or as the edit distance (also called Levenshtein distance or \ac{LER}). A pre-trained model for inference of English transcript can be downloaded, which achieves a \ac{WER} of just 6.5\%, which is close to what a human is able to recognize \parencite{mozillajourney}.

A model could be trained by providing training-, validation- and test-data for an arbitrary language (e.g. from the \textit{ReadyLingua} corpus). However, this is not the preferred procedure for this project for various reasons:

\begin{enumerate}
	\item The \textit{DeepSpeech} implementation was designed for \ac{ASR}. In such settings a low \ac{WER} is desirable. But this is not the main focus for this project. As a result, the architecture of the Mozilla implementation might be overly complicated for this project, although it might make sense for pure \ac{ASR} tasks. 
	\item The problem with above point is that more complex models usually require more training data. However, as for any neural network, the limiting factor for training a \ac{RNN} is often the lack of enough high quality training data. This becomes especially important when recordings in a minority language should be aligned. 
	\item The implementation requires an (optional) \ac{LM}, which is tightly integrated with the training process which might not be available for the target languages.
\end{enumerate}

For these reasons, the \ac{RNN} architecture of the \textit{DeepSpeech} model was used as a basis for a simplified version, which should (hopefully) require less training data in order to converge and still produce partial transcriptions that can be locally aligned.

\subsection{A simpler \textit{DeepSpeech} model}

An implementation of the \ac{RNN} used for \ac{STT} in the previous IP8 project was done in Python using Keras\footnote{\url{https://keras.io}}. The following simplifications and alterations were made:

\begin{itemize}
	\item No \ac{LM} 
	\item No convolution in first layer
	\item LSTM instead of SimpleRNN
\end{itemize}

Figure tbd. shows the architecture proposed in the \textit{DeepSpeech} compared to the simplified version used in this project / with the changes made for this project (eines von beidem verwenden).

tbd: Hier Bild Architektur einfügen

However, despite training on the \textit{LibriSpeech} corpus, this network did not seem converge. Furthermore, performance was a big issue, although the \ac{RNN} used a simpler architecture and no computational power was needed to query a \ac{LM}. Training on aligned speech segments from the \textit{LibriSpeech} corpus was not possible within project time because it would have taken approximately two months when using a single ac{GPU}. However, this duration is at least consistent with the experience made by the Machine Learning team at Mozilla Research, which used a cluster of 16 \acsp{GPU} that required about a week \parencite{mozillajourney} to train a variant \footnote{the variant used \ac{MFCC} as features whereas the original paper proposed raw spectrograms} of the \ac{RNN} originally proposed in \parencite{ctc_paper}.

In this project some experimentation was done to make the model converge:

\begin{itemize}
	\item increased number of \ac{MFCC} features from 13 to 26, because this is the value used in the \textit{DeepSpeech} model
	\item tried out different optimizers. Surprisingly, \ac{SGD} seemed to work better than Adam
	\item switched back to a Simple \ac{RNN} instead of \ac{LSTM}
	\item include a language model
\end{itemize}

\section{Including a \ac{LM}}

Although \ac{CTC} is the cost that is optimized during training, the main metrics to evaluate an \ac{ASR} system are usually \ac{WER} and \ac{LER}. The \ac{LER} is defined as the mean normalized edit distance ($ed(a, b)$ i.e. the number of insertions, deletions and changes required to produce string $b$ from string $a$) between a an inferred transcription (\textit{prediction}) and the actual transcription (\textit{ground truth} or \textit{label}). It operates on character level and is sometimes also referred to as \textit{Levensthein Distance}. The \ac{WER} builds upon the \ac{LER} but operates on word level, i.e. it represents the number of words in a inferred transcription, that need to be inserted, deleted or changed in order to arrive at the ground truth.

If a single evaluation metric is required, the \ac{WER} is often the better choice because it is more related to the way humans would assess the quality of a transcription: A transcription which might sound correct when read out loud but is full of spelling mistakes is not a good transcription. A \ac{LM} can help inferring orthographically correct words from sequences of characters detected by \ac{CTC} and hence decrease the \ac{WER}. Therefore, by using a \ac{LM} the quality of transcriptions improves perceivably, as the following example shows:

\begin{table}[!htbp]
	\centering
	\begin{tabular}{|l|l|r|}
		\hline
		\thead{transcript} & \thead{value} & \thead{\ac{LER}} \\
		\hline
		actual transcript & \code{and i put the vice president in charge of mission control} & $1.00$ \\ 
		\hline
		inference without LM & \code{ii put he bice president in charge of mission control} & $0.11$ \\ 
		\hline
		inference with LM & \code{i put the vice president in charge of mission control} & $0.08$ \\
		\hline
	\end{tabular}
	\caption{examples of infered transcripts with pre-trained DeepSpeech model with and without \ac{LM}\\(sample \code{20161203potusweeklyaddress} from the ReadyLingua corpus}
\end{table}

A \ac{LM} models probabilities of sequences of words. Therefore a \ac{LM} can be used to assign a probability to a sentence. A simple \ac{LM} that does that is the $n$-gram \ac{LM}. $n$-grams are overlapping tuples of words whose probability can be approximated by training on massive text corpora. Although a lot of research has been made in the field of using \ac{NN} for language modelling (like for machine translation), $n$-grams \ac{LM} are still widely used and often the right tool for many tasks \parencite{slp3}, because they are faster to train and require significantly less training data.

The Mozilla implementation includes an $n$-Gram \ac{LM} using \textit{KenLM}. The \ac{LM} is queried while decoding the numeric matrices produced by \ac{CTC} using \textit{Beam Search} or \textit{Best-Path} decoding. It uses a \textit{trie} and precompiled custom implementations in C of \textit{TensorFlow}-operations to maximize performance and dedicated weights for the influence the number of valid words and the \ac{LM} itself on the inferred transcription. It is therefore deeply baked in with the decoding process.

A simpler variant that is used in this project is to infer the transcriptions first with \textit{Beam Search} or \textit{Best-Path} decoding using the standard tools provided by Keras. The inferred transcriptions are then post-processed by running it through some sort of spell-checking, which is done as follows:

\begin{itemize}
	\item split the sentence into words 
	\item for each word $w_i$ in the sentence check the spelling by generating the set $C_i$ of possible corrections by looking it up in $V$, the vocabulary of the \ac{LM}, as follows:
	\begin{itemize}
		\item if $w_i \in V$ its spelling is already correct and $w_i$ is kept as the only possible correction, i.e.
		\begin{equation*}
			C_i = C_j^0 = \{ w_i \}
		\end{equation*}
		\item if $w_i \not\in V$ generate $C_i^1$ as the set of all possible words $w_i^1$ with $ed(w_i, w_i^1) = 1$. This is the combined set of all possible words with one character inserted, deleted or replaced. Keep the words from this combined set that appear in $V$, i.e.
		\begin{equation*}
			C_i = C_i^1 = \left \{ w_i^1 \mid (w_i, w_i^1) = 1 \land w_i^1 \in V \right \}
		\end{equation*}
		\item if $C_i^1 = \emptyset$ generate $C_i^2$ as the set of all possible words $w_i^2$ with $ed(w_i, w_i^2) = 2$. $C_i^2$ can be recursively calculated from $C_i^1$. Again only keep the words that appear in $V$, i.e.
		\begin{equation*}
			C_i = C_i^2 = \left \{ w_i^2 \mid ed(w_i, w_i^2) = 2 \land w_i^2 \in V) \right \}
		\end{equation*}
		\item if $C_i^2 = \emptyset$ keep $w_i$ as the only word, accepting that it might be either misspelled, a wrong word, gibberish or simply has never been seen by the \ac{LM}, i.e.
		\begin{equation*}
			C_i = C_i{>2} = \left \{ w_i  \right \}
		\end{equation*}
	\end{itemize}
	\item for each possible spelling in $C_i$ build the set $P$ of all possible 2-grams with the possible spellings in the next word as the cartesian product of all words, i.e.
	\begin{flalign*}
&	P = \{ (w_j, w_{j+1} | w_j \in C_j \land w_{j+1} \in C_{j+1} \}{,} && C_j \in \{C_i^0, C_i^1, C_i^2, C_i^{>2} \}
	\end{flalign*}
	\item score each 2-gram calculating the log-based probability using a pre-trained 2-gram-\ac{LM}
\end{itemize}

\section{Plotting a learning curve}

The total length of all transcribed speech segments in the \textit{LibriSpeech} corpus is roughly 47 days (1141 hours). Training on all these samples was not feasible within project time . However, we can get an estimation of the expected learning progress by plotting a \textit{learning curve}. This was done using transcribed audio segments from the \textit{LibriSpeech} corpus. Training was done on exponentially increasing amounts of training data (1, 10, 100 and 1000 minutes of transcribed audio). To assess the impact of using a \ac{LM}, the learning curve was plotted twice: Figure \ref{learning_curve_without_lm} shows the learning curve for \ac{WER} and \ac{LER} without using a \ac{LM}. Figure \ref{learning_curve_with_lm} shows the learning curve for training with a 4-gram \ac{LM}.

\begin{figure}
	\includegraphics[width=\linewidth]{./img/learning_curve_lm_beamsearch.png}
	\caption{Learning curve for training on 1/10/100/1000 minutes of transcribed audio from \textit{ReadyLingua} using a 2-gram \ac{LM} and Beam-Search decoding}
	\label{learning_curve_without_lm}
\end{figure}

\begin{figure}
	\includegraphics[width=\linewidth]{./img/learning_curve_lm_beamsearch.png}
	\caption{Learning curve for training on 1/10/100/1000 minutes of transcribed audio from \textit{ReadyLingua} using a 2-gram \ac{LM} and Beam-Search decoding}
	\label{learning_curve_with_lm}
\end{figure}

% To do:
% - train on LibriSpeech data
% - train using Best-Path
% - train using no LM
% - Reset calculation of WER/LER to original implementation